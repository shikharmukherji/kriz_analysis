\documentclass{article}
\usepackage[utf8]{inputenc}
\usepackage{amssymb}


\title{Kriz Analysis Chapter 1}
\author{Shikhar Mukherji}
\date{May 2021}

\setlength{\parindent}{0pt}
\begin{document}

\maketitle

\section*{Exercise 1}

Let $P$ be a non-zero polynomial with real coefficients of degree $n$. Then, by the Fundamental Theorem of Algebra, it has at least one complex root $c$. By Lemma $1.4.2$, if $c\in\mathbb{R}$, then $P$ may be written as $(x-c)Q$, where $Q$ is a polynomial with real coefficients of degree $n-1$. Similarly, if $c\notin\mathbb{R}$, then (by Proposition $1.4.4$) $P$ may be written as $(x-c)(x-\overline{c})Q$, where $Q$ is a polynomial with real coefficients of degree $n-2$. By writing $c$ as $a+bi$, with $a,b\in\mathbb{R}$, it is easily verified that $(x-c)(x-\overline{c})$ is a polynomial with real coefficients as well. Repeating this by induction with the polynomial $Q$ gives us the desired conclusion.

\section*{Exercise 2}

test

\section*{Exercise 3}
\section*{Exercise 4}
\section*{Exercise 5}
\section*{Exercise 6}
\section*{Exercise 7}
\section*{Exercise 8}
\section*{Exercise 9}
\section*{Exercise 10}
\section*{Exercise 11}
\section*{Exercise 12}
\section*{Exercise 13}

\end{document}
